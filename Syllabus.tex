\documentclass[11pt,twocolumn]{article}
\usepackage{hyperref,color}
\usepackage[margin=.5in]{geometry}
\usepackage{titlesec}
\usepackage{longtable}
\usepackage{gb4e}
%\usepackage[margin=.75in]{geometry}

\pagenumbering{gobble}

%paragraph formatting
\setlength{\parindent}{0pt}
\setlength{\parskip}{7pt}

\titlespacing\subsection{0in}{\parskip}{\parskip}
\titlespacing\section{0in}{\parskip}{\parskip}

%Just fill these in to fill in the basic syllabus information.
\newcommand{\coursename}{COSC-338}
\newcommand{\semester}{Spring 2024}
\newcommand{\classtimes}{M 610pm-730pm, T 610pm-730pm}
\newcommand{\myname}{Bill Six}
\newcommand{\myemail}{wesix@smcm.edu}
\newcommand{\university}{St. Mary's College Of Maryland}

\title{\coursename}
\author{{\university}---{\semester}---{\classtimes}}
\date{}


\begin{document}
\maketitle


\section{Instructor information}

\begin{tabular}{ll}
Name:&\myname \\
Contact:&\href{mailto:\myemail}{\myemail}\\
\end{tabular}

\section{Course description}

%Add new sections and content as needed.
Introduction to Computer Graphics, using OpenGL, both in Python and in C++.
At the completion of COSC-338, students will be able to:

\begin{enumerate}
  \item create graphical applications in 2D and in 3D
  \item apply coordinate-conversion transformations to objects
  \item control a virtual camera to traverse though the scene
  \item understand perspective projection, where objects further from the camera appear smaller
  \item add realism via color, lighting, and images
  \item control the application using mouse/keyboard/Game controllers, or GUI controls
  \item C628 - At the completion of COSC338, students will be able to explain creation of computer graphics by answers to examination questions.
  \item C629 - At the completion of COSC338, students will be able to produce an computer-generated image as demonstrated by which shows proper 3D shading.
  \item C630 - At the completion of COSC338, students will be able to produce an computer-generated image as demonstrated by which shows proper 3D reflections.
  \item C631 - At the completion of COSC338, students will be able to produce an computer generated image as demonstrated by which accurately depicts a light-source and its effects on the objects.
  \item C632 - At the completion of COSC338, students will be able to use OpenGL as demonstrated by creating a computer-generated image.
\end{enumerate}

The course
begins with an easy to understand version of OpenGL, version 2, and progresses
to version 3.3+, which allows the programmer to create much more realism
in the graphics.

\section{Grading}

\begin{center}
\begin{tabular}{cc}
\begin{tabular}{l|l}	%For grade items (quizzes, homework, etc.)
Item&Percent\\\hline\hline
Homework&30\\
Midterm&20\\
Final&20\\
Projects&20\\
Paper&10
\end{tabular}
&
\begin{tabular}{ll}

\end{tabular}
\end{tabular}
\end{center}

\section{Schedule}

TBD based off of feedback.  We will cover:


\begin{enumerate}
\item{modelviewprojection. github.com/billsix/modelviewprojection.  Covers the basics of
  OpenGL, and how to think about placing objects in a scene, how to place a camera, and
how to use basic inputs from the keyboard.  The duration for this material will be the first month or two, and is in Python.}
\item{After that, we will cover https://learnopengl.com, where the C++ code is at https://github.com/billsix/LearnOpenGL.}
\item{Colors, Materials, and Lighting}
\item{Texturing}
\item{Shader language.  Vertex Shaders}
\item{Vertex Shaders, Fragment Shaders.}
\item{Misc topics of interest}
\item{This class puts an emphasis on reading code and modifying code; not writing everything from scratch.
  There will be small projects to demonstrate understanding of the material.}
\item{There will be a project in which you study the source of a project of your choice,
  and write a paper about what you learned, and/or modify it to do something different.
  The instructor will provide examples that could be suitable for study.
  }

\end{enumerate}

\section{Software/OS}


Modelviewprojection requires Python3 to be installed, and works on Linux/Windows/macOS. Spyder
is a decent Integrated Development Environment that students may choose to use to develop and
run code.
On Windows, students could also Visual Studio Community, and install the Python Extension tools.
On Mac, install Python through the regular installer, or anaconda, or through macports or homebrew.
On Linux, use the package manager.





\end{document}
